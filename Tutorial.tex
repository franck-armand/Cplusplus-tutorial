\documentclass[11pt, a4paper]{article}
%\DeclareUnicodeCharacter{200A}{!!!FIX ME!!!}

\usepackage[utf8]{inputenc}
\usepackage[margin=1in]{geometry}
\usepackage{txfonts}
\usepackage{blindtext}
\usepackage{listings}
\usepackage{hyperref}
\usepackage{graphicx}
\usepackage{xcolor}
\usepackage{cleveref}

\newsavebox{\mybox}
\savebox{\mybox}[0.5\textwidth][c]{aaaa}
%\renewcommand{\lstlisting}{code}
\renewcommand{\lstlistlistingname}{List of codes}
\lstdefinestyle{chstyle}{%
backgroundcolor=\color{gray!12},
basicstyle=\ttfamily\small,
commentstyle=\color{green!60!black},
keywordstyle=\color{magenta},
stringstyle=\color{blue!50!red},
showstringspaces=false,
captionpos = b,
numbers = left,
numberstyle=\footnotesize\color{gray},
numbersep=10pt,
stepnumber=2,
tabsize = 3,
frame=L,
framerule=1pt,
rulecolor=\color{red},
breaklines = true,}


\title{C++ Tutorials for Beginners}
\author{Kouassi Franck Armand Prince
\date{09.05.2021}
}


\begin{document}

\maketitle\hrule
% \begin{center}
% \textbf{C++ Tutorials for Beginners}
% \end{center}
\newpage

\tableofcontents
\listoffigures
\lstlistoflistings
\pagebreak

\section{Getting Started}
%{\blindtext}
This section introduces the basics of C++ programming language and the tools
needed to follow this tutorial. The goal of this tutorial is to help beginners
getting started with C++ programming language. It does not require you to you to
have a prior programming background kownledge. All you have to do is to follow along
with me and and try to \textbf{\textit{WRITE}} the code on your own machine not just read.
Believe me it is easier to read and assume that you have mastered it until you are required
to write the code by yourself, that is where it realize you have not properly understood it.

\subsection{Understand the Computer Language}
%{\blindtext}
Your computer is an incredible and complicated device. Basically, the computer
understands one simple language composed of 0 and 1. Thus a message like this
"\textit{01001100101001010}" could mean "open a window" for instance. Fortunately, we do not have
to learn this language (Binary language). Programmers created languages which are much simpler
than binary language. Here you could check \href{https://en.wikipedia.org/wiki/List_of_programming_languages}
{\emph{\textit{the number of programming languages.}}} \newline
All programming languages have the same goal, that is being able to easily and efficently
communicate with the computer compare to binary language.\newline Here, is how it works :

1. You write the instructions to be executed by the computer in a programming language (e.g C++)

2. The instructions are translated in binary (0 and 1), the language understood by the computer.

3. The computer can now decode the message and executes your request.

\begin{figure}[!ht]
\includegraphics[width = \linewidth]{diagrams/compling.png}
\caption{Compliling Process.}
\label{fig:Compliling Process}
\end{figure}

\subsection{C++ against other programming languages}
Before we start talking about why C++ reprsents a powerful language despite its age. let's discuss the
the key points to analyze before diving into a language.

There exists numerous programming languages as mentioned in above section, although some languages are interesting, 
they are seldom used. The main challenge that comes with these languges, is that they do not have a very big community 
so imagine you working on a project and you are facing a problem, it is difficult to find help since not so many people 
are using the the language.This explains why C++ represents a good choice for debutant programmers. You are not alone, a lot
ressources are availble to guide through your learning process, also C++ is still being widely used.

Another interesting aspect to look at as well is the programming language level. There are of two (02)
types: \textbf{\textit{{high level}}} and \textbf{\textit{{low level}}}.
\newline
\textbf{\textit{{high level}}}: is a language that is that is far from binary language and really
to humans languge,it allows to easily understand and translate instructions contrary to \textbf{\textit{{low level}}}
which a language closed to machine language and generally requires much more effort but gives you more 
control over what you can do, it is a trade-off.

\begin{figure}[h!]
    \includegraphics[width = \linewidth]{diagrams/low vs high.png}
    \caption{Programming Language by level.}
    \label{fig:Programming Language by level}
\end{figure}
C++ is a low level language. Do not panic, although coding in C++ might be a little complex, you will 
have in your possession a very \textbf{\textit{{powerful}}} and particulary \textbf{\textit{{fast}}} language.
Infact, if most games are developed in C++, it is because it is the language capable of coupling
speed and power, that makes it an essential language.



\subsection{Summary of C++}
Here we are going to showcase some aspects of C++ that make it an important language regardless of
how long it has been since it creation.
\begin{itemize}
    \item \textbf{Popularity} : C++ is one of the most popular
    languages in the world. It is used by some 4.4 million developers worldwide
    \item \textbf{Large Community} : There is a large online community
    of C++ users and experts that is particularly helpful in case any support is required.
    There is a lot of resources available on the internet regarding C++.
    \item \textbf{Portable} : Programs developed in C++ can be moved from one platform
    to another. This is one of the main reasons that applications requiring multi-platform
    or multi-device development often use C++.
    \item  \textbf{Speed} : Programs written in C++ language execute more faster compare
    to most programming languages\end{itemize}

\paragraph{Snippet of C++}
To give you an idea of how the code looks, let's look at a simple C++ program displaying "Hello world!"
on the screen. Do not try to understand the code just appreciate the beauty and structure. We will
go into details in the following sections

\begin{lstlisting}[caption=Sample example of C++ programming language, style=chstyle,
    language=C++, label = c-sample]
//=================================================
// Sample example of C++ programming language
//=================================================

#include <iostream>

using namespace std;

int main()
{
    cout << "Hello world!" << endl;
    return 0;
}
//=================================================
\end{lstlisting}
\textit{ If you are interested in knowing the story of C++ starting from its creation,
\href{https://en.wikipedia.org/wiki/Bjarne_Stroustrup}{you can learn all about C++ from wikipedia}}

\subsection{Summary}
% \subparagraph{sub paragraph}
% {\blindtext}
\begin{itemize}
    \item Programs allow us to efficently control actions on the computer: web browsing, text editing etc
    \item In order to create a program, we write instructions for the computer using a programming; source code
    \item The source code must be converted in binary by what we could a compiler, it allows the executation 
    of the code.
    \item  C++ is a widely used programming language, it is an evolution of C programming due to the fact that 
    it allows Object Oriented Programming (OOP), a very powerful programming feature.
\end{itemize}
\newpage

\section{Environment setup}
In this section, we are going to introduce the tools needed to follow
this tutorial.From our previous discuss, you already know by now an important
tool needed, Yes you are right, you need a Compiler, the program that
converts your C++ code into the computer readable format.

Aside these, there are additional tools needed for you to code with ease
\begin{itemize}
    \item \textbf{A text editor}: It will allow you to write your source
    code. On windows we have Notepad or Vi on linux. But of course, it is
    less recommed because as your code gets bigger and bigger, you might not 
    not be  able to fully control it.
    \item \textbf{A compiler}: as mentioned above, it converts your source code into
    binary format for the computer
    \item \textbf{A debugger}: it helps you find bugs in your programs.
\end{itemize}
From now on we have two options (02) either we get the programs seperatly
which is of course much complicated, but on Linux most programmers prefer
to use them in that way, I will not go into much details here, instead we
are going to explore the simple way. We can get a program "3 in 1",
Yes you heard me correctly, a tool capable of handling the 3 listed tools
It is commonly refered to as an \textbf{IDE} (Integrated Developement Environment).
There are of numerous types. In this tutorial we are not going to discuss
their similarities. I personnaly recommed Visual Studio Code and here is
how you can \href{https://code.visualstudio.com/docs/languages/cpp}
{get started with C++ for Visual Studio Code}. So go ahead and install the
necessary packages.

\section{Your first C++ code}
The “Hello World” program is the first but most vital step towards
learning any programming language and it is certainly the simplest
program you will learn with each programming language. All you need
to do is display the message "Hello World" on the output screen.
\lstinputlisting[language=c++, style=chstyle, label=first C++ code, caption=First C++ code]{codes/first.c++}

\noindent \textbf{Output:}
\lstinputlisting[style=chstyle, label= hello world, caption=First C++ code Output]{codes/1.txt}

\noindent
\textbf{1}. \fbox{\color{green!60!black}// Your first C++ program }\\
In C++, any line starting with \fbox{\color{green!60!black}//} is a comment. Comments are
intended for the person reading the code to better understand the
functionality of the program. It is completely ignored by the C++
compiler.

\noindent
\textbf{2}. \fbox{\color{magenta}\#include \color{black}\textless iostream\textgreater}\\
The \#include is a preprocessor directive used to include files in our
program. The above code is including the contents of the iostream file.

\noindent
\textbf{3}. \fbox{\color{magenta}int \color{black}main() \{...\}}\\
A valid C++ program must have the main() function. The curly braces
indicate the start and the end of the function. \\ The execution of
code beings from this function.

\noindent
\textbf{4}. \fbox{std::cout \textless\textless \color{blue!50!red}"Hello World!";}\\
std::cout prints the content inside the quotation marks.
It must be followed by '\textless\textless'  followed by the format string.
In our example, "Hello World!" is the format string. \\
\textbf{Note: } \fbox{;} is used to indicate the end of a statement.

\noindent
\textbf{5}. \fbox{\color{magenta}return \color{black} 0;}\\
The return 0; statement is the "Exit status" of the program.
In simple terms, the program ends with this statement.

As you might have noticed, the code in \ref{c-sample} looks a little different from the one
in \ref{first C++ code} but produce the same output do not worry we are going to get to the
difference soon. \\
\noindent In \ref{c-sample}, line 7 we used \fbox{\color{purple!85}using namespace \color{black} std;}
to tell the compiler to use standard namespace. Namespace collects identifiers used for class, object
and variables. Name space can be used by two ways in a program, either by the use of using statement
at the beginning, like we did in above mentioned program or by using name of namespace as prefix
before the identifier with scope resolution (::) operator. Thus with namespace std,
\fbox{std::cout \color{black}\textless\textless \color{blue!50!red}"Hello World!";}
becomes \fbox{cout \textless\textless \color{blue!50!red}"Hello World!";}
like in our previous example, much simpler right !! it is totally up to you, to define on which style
suits you the most.\\
\noindent  also, the keyword \fbox{endl} is used to denote the end of a line therefore, the next line 
of code will be printed on a new line.\\
It is also important to notice that you could combine instructions into a single one. for example,
\lstinputlisting[style=chstyle, language=C++, label=Compress C++ code, caption= Compress C++ code ]
{codes/oneLine.c++}
This code output the two (02) instructions on 02 lines, you can run the code and see the output.

\subsection{Make your code more readable}
In order to allow others and yoursel to understand your code, it is recommended to add
comments to your code. Now we are going to learn how to comment our program. We introduced
the concept of the in the previous section but lets dive depper into it.\\
There exists two (02) types of comments :
\begin{itemize}
\item {\textbf{Short comments}}: as stated in the names they are short and can be written
    in one line of code. To write a short comment, you just need to start with \fbox{\color{green!60!black}//}
    following by your comment.
\end{itemize}
\lstinputlisting[style=chstyle, language=C++, label= First short commment, caption= First short commment]
{codes/comment1.txt}


\begin{itemize}
\item {\textbf{Long comments}}: if your comments are long enough and can not fit in one line,
    you can have a block comment. You just need to start and end your block comment
    with \fbox{/*}.
\end{itemize}

\lstinputlisting[style=chstyle, language=C++, label= First block commment,
caption= First short commment]{codes/comment2.txt}

\noindent Generally, we do not write too much in the comment section, just the necessary
information, unless you have to.

\begin{itemize}
    \item {\textbf{Let's comment our code}}
\end{itemize}
\lstinputlisting[
    style=chstyle, language=C++, label= Comments in C++,
    caption= Comments in C++
]{codes/helloworldCommented.c++}

\noindent After you run this code, nothing will change, the output will still be the same
because comments are simply ignored by the compiler.

\subsection{Summary}
\begin{itemize}
    \item The execution of code begins from the main() function. This function is mandatory.
    This is a valid C++ program that does nothing.
    \item The \fbox{cout} is used to display a message.
    \item You can comment your codes in two (02) ways
    \fbox{// Comment} or \fbox{/* Comment */}
\end{itemize}


\end{document}